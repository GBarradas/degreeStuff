\documentclass[letterpaper,12pt]{article}
\usepackage{graphicx}
\graphicspath{ {./images/} }
\usepackage{graphicx} % takes care of graphic including machinery
\usepackage[margin=1in,letterpaper]{geometry} 
%++++++++++++++++++++++++++++++++++++++++

\title{\includegraphics[width=0.5\textwidth]{imagens/índice.png}
\\ [0.9cm] Editor de Texto "ed"}
\author{André Baião 48092\\Gonçalo Barradas 48402}
\date{junho, 2021}
\maketitle
\vspace{1.5cm}
\begin{center}
  U.C. Arquitectura \ de \ Computadores\\Licenciatura \ em \ Engenharia \ Informática  
\end{center}
\vspace{3.0 cm}
\begin{center}
    \textbf{Docentes}\\   Miguel \ Barão\\      Pedro \ Salgueiro
\end{center}

\newpage
\begin{document}
\section{Introdução}
\onehalfspacing 
\large 
Para avaliar o conhecimento e o domínio em \emph{RISC-V} adquirido ao longo do semestre par, foi-nos proposto elaborar um código em \emph{RISC-V} para simular o editor de texto "ed", editor de texto padrão do Unix. Este editor de texto pode ser testado num terminal de um computador (S.O Linux por exemplo).\\
O projecto deve elaborar as funções do editor de texto como a função \emph{insert} (comando i) que insere texto antes da linha actual, a função \emph{append} (comando a) que insere texto na linha actual, a função \emph{change} (comando c) que substitui a linha actual por novo texto entre outras, e foi dividido em duas fases.\newline
A primeira fase deste projecto foi a elaboração de um código em linguagem c, para auxiliar a realização do código em \emph{Assembly}.
A segunda fase do projecto consiste na elaboração do código em \emph{Assembly} utilizado \emph{RISC-V}.
\vspace{1,5cm}


\section{Metodologia}

Como mencionado acima o projecto foi dividido em duas fases, a primeira consistia em elaborar um código em linguagem c e a segunda um código em \emph{Assembly RISC-V}.
Para realizar o código em linguagem c, foi necessário avaliar e definir a estrutura de dados que mais se adequava ao problema em questão. Desta maneira foi decidido elaborar o código utilizando "\emph{Doubly Linked Lists}". A segunda opção seria utilizar \emph{Arrays}, mas apesar desta forma ser menos propensa a erros optámos por utilizar as \emph{Doubly Linked Lists} com o objectivo de poder dinamizar o projecto de forma eficiente.
Foi assim elaborado o código em "c", com todas as funções necessárias para simular este editor de texto, para que pudéssemos proceder para a realização do código em \emph{Assembly} com o auxílio de uma linguagem de alto nível.\vspace{1,5cm}

\newpage
\section{Desenvolvimento}

Nesta fase do projecto começámos por tentar perceber como simular apontadores, visto que não é possível utilizá-los como na linguagem "c", e com o desenvolver das funções, reparou-se que não seguíamos o código em c\newline realizado e que seguiríamos por desenvolver o código em \emph{assembly} de forma autónoma.\\
Foram estabelecidos no início deste projecto alguns registos de função para guardar argumentos:\\
\textbf{a1} == Guarda a posição final do texto.\\
\textbf{a2} == \emph{"SAFE"} Este registo guarda 1 ou 0 caso o ficheiro esteja guardado ou não, é necessário para a função "q" que só sai do editor se o ficheiro estiver guardado.\\
\textbf{a3} == Este registo guarda o número de linhas, isto é, conta as linhas já inseridas e por si conta também quando necessário a posição a inserir ou a apagar do ficheiro.\\
\textbf{a4} == Neste registo vai ficar guardada a posição a ser inserido ou eliminado o texto (dependendo da instrução), assim este registo pode ser comparado com "a3".\\ 
Serão utilizados outros registos durante a elaboração do código mas serão registos de memória temporária.\\
Contudo, com alguma pesquisa, e auxílio das ferramentas fornecidas pelos professores percebemos que teríamos de criar um espaço na \emph{heap} pois esta é uma memória dinâmica que cria espaço conforme o necessário, tal como os nós das nossas listas em "c".\vspace{0.5cm}\\
A função \emph{\textbf{"inicializelist"}} é a função responsável pela criação de espaço e que torna possível colocar informação nesse espaço de memória, como a criação de um novo nó de uma lista.
Para isso damos a instrução ao simulador para reservar um  espaço na memória dinâmica \emph{heap} \vspace{0.5cm}\\
A função \emph{\textbf{"insert"}} está dividida em duas funções distintas. Isto deve-se à     existência de um comando que insere texto no meio de linhas anteriormente inseridas.\\
Estas funções são a \emph{\textbf{"insert\_end"}} e a função \emph{\textbf{"insertmidle"}}.\\
A função \emph{\textbf{"insert\_end"}} é responsável por inserir um novo texto no fim do texto. Para isto instruímos o processador a ler uma string, e, com a ajuda de uma função auxiliar \emph{"count"} conta o tamanho da string inserida, introduzindo-a na \emph{heap} utilizando o espaço necessário.\vspace{0.5cm}\\
A função \emph{\textbf{"insertmidle"}} é um pouco mais complexa, pois tem de procurar a posição a inserir o novo texto e criar o espaço necessário para a sua insersão.\vspace{0.5cm}\\
Tal como a função \emph{\textbf{"insert"}} a função \emph{\textbf{"delete"}} também está dividida em duas funções distintas pelo mesmo motivo, a existência de um comando para apagar o último texto inserido ou texto já inserido anteriormente.\vspace{0.3cm}\\\
Uma destas funções é a função \emph{\textbf{"delete\_last"}} que é a responsável por\newline eliminar a última linha de texto inserida,  para isso começa no final do texto e vai percorrendo para trás até encontrar o caractere de final de linha "\textbackslash n" e subtrai ao número de linhas o valor 1. \vspace{0.5cm}\\\
A função que corresponde à eliminação de uma linha do texto chama-se \emph{\textbf{"delete\_midle"}}, e tal como a função \emph{\textbf{"insertmidle"}}, esta também tem de percorrer o texto até chegar à posição onde começa a linha a eliminar e depois calcula o tamanho da string e coloca o restante  texto na posição da string eliminada e subtrai ao numero de linhas 1.\vspace{0.5cm}\\\
De seguida foram desenvolvidas as funções \emph{\textbf{"print"}} e \emph{\textbf{"printall"}}.\\
A \emph{\textbf{"print"}} carrega o endereço da \emph{heap} para "a0" (inicio do texto) e depois percorre ate ao inicio da linha pretendida, depois dá ao processador a\newline instrução para imprimir os caracteres em \emph{"ASCII"}.\vspace{0.5cm}\\
A função \emph{\textbf{"printall"}} funciona de uma forma semelhante, mas começa no inicio do texto e vai percorrendo todo o texto e imprimindo linha a linha até chegar ao final\vspace{0.5cm}\\
As funções desenvolvidas seguidamente foram as funções \emph{quit}. Estas funções têm como objectivo sair do editor de texto, de duas maneiras diferentes pois uma da funções sai do editor mesmo que o ficheiro editado não esteja guardado e a outra apenas sai do editor se o ficheiro estiver guardado.\vspace{0.5cm}\\
A primeira \emph{\textbf{"quitQ"}} apenas contém a instrução para fechar o programa.\vspace{0.5cm}\\\
A segunda \emph{\textbf{"quit"}} faz a comparação entre o nosso valor \emph{"SAFE"} (a2) para saber se o texto esta guardado  caso o texto esteja guardado num ficheiro saltamos para a função \emph{\textbf{"quitQ"}}, caso contrário mostra uma mensagem com um ponto de interrogação e não fecha o programa.\vspace{0.5cm}\\\
Após todas estas funções concretizarem o desejado procedemos para o\newline desenvolvimento do código para a leitura e analise dos comandos.\vspace{0.5cm}\\
A função \textbf{comando} dá a instrução ao processador para ler uma \emph{string}, após essa leitura vai analisar a string e determinar qual a instrução a realizar. Caso não reconheça a instrução mostra uma mensagem com um ponto de interrogação .\vspace{0.5cm}\\
Instrução\textbf{ "i"}: esta instrução pode ser chamada de duas maneiras. Quando o comando é apenas "i" insere texto no fim do documento. Quando é dado o comando i, mas com um número agregado \textbf{(5i)} é necessário utilizar a função \emph{"atoi"} que detecta qual o número agregado à instrução e assim insere na linha desejada.\vspace{0.5cm} \\
A instrução \textbf{"a"} tal como a instrução \emph{"i"} também pode ser chamada de duas maneiras. No caso de ser apenas "a" vai inserir no fim do documento. Quando com um número agregado \textbf{(5a)} vai inserir na linha a seguir à linha indicada na chamada da instrução.\vspace{0.5cm}\\
Instrução \textbf{"c"} é outra instrução semelhante às instruções anteriores, pois também se pode chamar de duas maneiras. Apenas \textbf{"c"}, eliminando a última linha e insere numa nova última linha, ou agregado a um número \textbf{"5c"} e assim a função vai eliminar a linha em questão e começar a inserir nessa mesma linha.\vspace{0.5cm}\\
A instrução \textbf{"d"} é um pouco diferente das outra funções, pois pode ser chamada de três maneiras diferentes. Tal como as funções acima mencionadas quando é apenas um "d" a última linha é eliminada, agregado a um número (5d) a linha pretendida é eliminada, mas acrescenta-se uma nova maneira de chamar esta função para que opere num intervalo (2,4d ou 4,\$d (da linha 2 á linha 4)) neste caso o intervalo em questão vai ser eliminado. Caso o segundo valor do intervalo seja um \textbf{"\$"} vão ser eliminadas todas as linhas a partir da linha lida até ao fim do texto.\vspace{0.5cm}\\
A instrução \textbf{"p"} é igual à função \textbf{"d"} só que em vez de eliminar, imprime a linha ou conjunto de linhas desejadas. No entanto, esta função tem ainda mais uma forma de ser chamada pois se a instrução for \textbf{"\%p"} todo o texto deverá ser impresso.\vspace{0.5cm}\\
Instrução \textbf{"q"} salta para a função \textbf{"quit"}\vspace{0.5cm}\\
Instrução \textbf{"Q"} salta para a função \textbf{"quitQ"}.\vspace{0.5cm}\\
A instrução \textbf{"e"} é chamada seguida do nome do ficheiro (e file\_name) e deverá abrir e ler o respectivo ficheiro para a \textbf{heap}.\vspace{0.5cm}\\
Instrução \textbf{"f"} deve ser chamada seguida do nome do ficheiro a criar (f file\_name) e após a sua chamada é criado um ficheiro com o nome pretendido e deve ser impresso o nome do ficheiro na consola.\vspace{0.5cm}\\
Por fim a instrução \textbf{"w"} é responsável por escrever texto num ficheiro seja este obtido pela instrução \textbf{"e"} ou criado pela instrução \textbf{"f"}. No caso de não existir um ficheiro aberto vai ser mostrada uma mensagem com um ponto de interrogação na consola.\\

\newpage
\section{Discussão de Resultados}

Os resultados obtidos deste projecto foram os esperados contudo existiram alguns obstáculos difíceis de ultrapassar na realização deste código.
Facilmente conseguimos obter todas as funções a concretizar o seu propósito individualmente, e foi quando testámos várias funções em simultâneo que começaram a aparecer erros, e por isso, foi necessário uma análise\newline detalhada da execução das nossas funções e severas remodelações ao nosso código inicial de forma a que todas as funções trabalhassem em harmonia sem se prejudicarem umas às outras.\vspace{0.5cm}\\
O primeiro erro apareceu quando testámos a função \textbf{"delete\_last"} e de seguida a função \textbf{"insert\_end"}, que após apagar a última linha inserida, e ao inserir uma nova linha, a mesma era inserida na linha a seguir á linha eliminada. Isto ocorria porque, após a linha ser eliminada o endereço que continha a posição final permanecia inalterado.\vspace{0.5cm}\\
O segundo erro aconteceu quando testámos a função \textbf{"printall"} após a função \textbf{"delete\_last"}, quando todo o texto era impresso, também era impressa a linha que supostamente foi eliminada. Este erro foi fácil de resolver uma vez que após a execução da função \textbf{"delete\_last"}, reparámos que o número de linhas não se alterava.\vspace{0.5cm}\\
As funções que nos deram mais problemas foram funções relacionadas com ficheiros, \textbf{"open\_file" \textbf{(e)}} \textbf{"write\_file"\textbf{(w)}} \textbf{"define\_name"\textbf{(f)}}, pois foram funções mais complicadas de perceber a sua implementação.
\\
\newpage
\section{Conclusão}
Com a realização deste trabalho compreendemos que a utilização de uma linguagem de programação de baixo nível, é muitas vezes, limitada nos seus recursos e ferramentas para que possamos desenvolver um código eficaz. No entanto com bastante prática é possível dominar a utilização destas linguagens.\\
A posição das instruções e das label's afectam bastante o código e é um dos pontos a ter bastante atenção.\\
É necessário também ter bastante atenção com os registos utilizados, pois uma má utilização dos registos, principalmente com registos de função,\newline pode-se perder informação e danificar o código.\\
Para uma correcta utilização de ficheiros neste programa seria necessário uma abordagem mais detalhada sobre o próprio tema, pois foi onde sentimos as maiores dificuldades na implementação do código.\\
O projecto em sim foi interessante e ajudou bastante a aperfeiçoar os\newline conhecimentos adquiridos ao longo deste semestre e ainda adquirir novos\newline conhecimentos na utilização deste tipo de linguagem. 

%++++++++++++++++++++++++++++++++++++++++

\section{Bibliografia}
Barão, Miguel and Salgueiro, Pedro. 2021. "Aulas de Arquitectura de Computadores I" In. Universidade de Évora.\\

\end{document}

