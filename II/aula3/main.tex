 \documentclass{article}
\usepackage[portuguese]{babel}
\usepackage[utf8]{inputenc}
\usepackage[a4paper, total={16cm, 24cm}]{geometry}


%---------------------------------------------------------------%
\title{\textbf{Armazenamento genético de dados}}
\author{André Baião, Gonçalo Barradas, Guilherme Grilo\\
48092, 48402, 48921}
\date{Março de 2022}
%---------------------------------------------------------------%
\begin{document}
\maketitle
\begin{abstract}
Vivemos numa era cada vez mais digital, em que praticamente tudo o que fazemos gera dados, dados esses que têm de ser armazenados.

A produção de dados digitais cresce de forma exponencial, superando atualmente a produção e evolução das formas de os armazenar. O armazenamento convencional inclui armazenamento magnético (HDD's), ópticos (Blu-ray) e sólidos (SSD's). É por isso essencial encontrar novos métodos que sejam seguros para armazenamento de dados (\cite{Ceze2019}; \cite{STANLEY2020}).

O método de armazenamento genético de dados, mais concretamente em DNA, está atualmente em estudo, havendo já alguns avanços no seu desenvolvimento. O processo base consiste em converter a informação digital que se encontra armazenada em código binário (0,1) para a forma de organização do DNA “código de bases azotadas” (A, T, C, G) (\cite{YOO2021}), existem varias hipóteses de codificação que já foram exploradas.

As principais vantagens na utilização do DNA são, a sua estabilidade, longevidade, densidade e capacidade de cópia. A sua longevidade é de aproximadamente 500 anos em condições normais de armazenamento, comparativamente com outros meios de armazenamento de dados que tem uma vida útil de apenas algumas décadas. O DNA tem capacidade de armazenar 1018 bytes em apenas um milímetro cúbico, o que representa uma densidade de armazenamento 6 vezes maior do que qualquer outro meio de armazenamento já utilizado, pode ainda ser copiado de forma exponencial, o que melhora significativamente a eficiência de backups de dados. \cite{Cox2001}; \cite{STANLEY2020})

Desta forma, o armazenamento em DNA parece ser o candidato mais atraente para substituir os atuais meios de armazenamento de dados (\cite{YOO2021}).
\end{abstract}

\bibliography{ref.bib}
\bibliographystyle{plain}

\end{document}